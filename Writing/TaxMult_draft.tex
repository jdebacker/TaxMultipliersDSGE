\documentclass[article,11pt,letterpaper,fleqn]{article}
\usepackage{graphicx,color}
\usepackage{array}
\usepackage{threeparttable}
\usepackage[format=hang,font=normalsize,labelfont=bf]{caption}
\usepackage{colortbl}
\usepackage{multirow}
\usepackage{geometry}
\usepackage{subfigure}
\geometry{letterpaper,tmargin=1in,bmargin=1in,lmargin=1.25in,rmargin=1.25in}
\usepackage{hyperref}
\hypersetup{colorlinks,%
citecolor=red,%
filecolor=red,%
linkcolor=red,%
urlcolor=blue,%
pdftex}
\usepackage{amsmath}
\usepackage{amssymb}
\usepackage{amsthm}
\usepackage{harvard}
\usepackage{setspace}
\usepackage{float,graphicx,color}
\usepackage{appendix}
\usepackage{longtable}
\newtheorem*{thm}{Theorem}
\theoremstyle{definition}
\numberwithin{equation}{section}
\newcommand{\cn}{\citeasnoun} % shortens command to cite as noun
\newcommand\ve{\varepsilon}



\author{Jason DeBacker\thanks{Tel: 770-289-0340, Email: \href{mailto: jason.debacker@gmail.com}{jason.debacker@gmail.com}} and Richard W. Evans\thanks{Center for Growth and Opportunity, Email: \href{mailto: revans@thecgo.org}{revans@thecgo.org }}}
\title{Tax Multipliers in a DSGE Model\thanks{Preliminary-- Please do not cite without author's permission.}}
\date{First Draft: September 28, 2010\\ This Version: \today}


% make tables with smaller sized font
\makeatletter
\def\table{\@ifnextchar[{\table@i}{\table@i[\fps@table]}}
\def\table@i[#1]{\@float{table}[#1]\footnotesize}
\makeatother



%\setlength{\topmargin}{-0.4in}
%\setlength{\topskip}{0.3in}    % between header and text
%\setlength{\textheight}{9.0in} % height of main text
%\setlength{\textwidth}{6in}    % width of text
%\setlength{\oddsidemargin}{39pt} %even side margin
%\setlength{\evensidemargin}{39pt} %odd side margin

\begin{document}
\bibliographystyle{aer}
\maketitle



\begin{abstract}
This paper analyzes the stabilization properties of tax cuts in a DSGE model.
\end{abstract}






%\section{Introduction}
%\label{sec:intro}


Consumption taxes have long been favored by tax economists for their relative efficiency vis-a-vis taxes on capital and labor.  This paper attempts to argue that consumption taxes should also be favored by macroeconomists for their stabilization properties.  Cuts to consumption taxes can quickly and effectively stimulate aggregate demand. The following analysis studies the quantitative impact of tax cuts in a dynamic schocastic general equilibrium (DSGE) model.  In addition to consumption taxes, the model nests other tax policies, such as investment tax credits and changes to expensing and depreciation policies.  These policies target investment by lowering the after-tax cost of capital.  Historically, such instruments have been used by the U.S. to provide counter-cyclical fiscal policy (e.g. the Economic Recovery Act of 1981 increased the ability of businesses to take advantage of the investment tax credit).  Despite their use, such policies have not been studied in a DSGE environment.

The goal of this paper is to study specific tax policies in a DSGE environment.  In a sense, bring in the tax policies favored by tax economists into a model used by macroeconomists.  Recently, a number of researchers have used DSGE models to study the government spending multiplier (e.g., \cn{CER2010}), but less work has been done on tax multipliers.  Those who provide insights on tax multipliers (e.g., \cn{Zubairy2010}, \cn{IMF2010}) focus on multipliers associated with broad taxes on capital and labor income.  Such tax cuts are generally not the most effective counter-cyclical policies.  We are the first to provide quantitative insights into the the size of the multipliers associated with tax cuts that directly target consumption and investment.

Two recent, but related, events have spurred interest in consumption taxes; the 2007-2009 recession and the large federal budget deficits.  In a 2008 \emph{Financial Times} aricle \cn{LK_FT2008} argue that a national sales tax holiday would be an especially effective response to the dip in aggregate demand experienced during the 2007-2009 financial crisis.  While possible, it is difficult to implement such a policy without a national sales tax.  Recent debate about the fiscal sustainability of the federal government has brough much attention to value added taxes (VAT) as a way to increase revenue with minimal efficiency costs.  We hope to show that a tax on consumption has a second positive attribute - a non-zero tax on consumption allows policy makers important flexibility for macro stabilization policies.


Maybe cite some of the micro evidence of effects of sales tax holidays...

Stress:
\begin{itemize}
\item lower dwl due to consumption tax than other forms of tax
\item faster impact of consumption tax cut
\item consumption beats spending because people get to choose what to spend it on (though we can't easily model this)
\item current focus on the VAT, which is a consumption tax.  Most of talk is about less distortionary way to raise revenue, but it would also allow to a good fiscal stabilizer.
\end{itemize}
Other novelties:
\begin{itemize}
\item we build a model that can handle other tax cuts like bonus depreciation (something simple so don't have to take account of vintage) and investment tax credits- see what the multiplier is for these in short and long run
\item Be best if could have heterogenous firms who vary in productivity and have price stickiness.  Build in corporate income tax, div tax, cap gains tax in more realistic detail (now all lumped in tax on capital).
\end{itemize}

\section{Environment}
\label{sec:model}

\subsection{Households}
Representative household.

HH maximizes:
\begin{equation}
U=E_{0}\sum_{t=0}^{\infty}\beta^{t}u(c_{t},l_{t},g_{t})
\end{equation}

subject to the real budget constraint:
\begin{equation}
\begin{split}
(1+\tau_{t}^{c})c_{t}+\frac{b_{t}}{P_{t}}+i_{t}= & \frac{(1+r_{t}(1-\tau_{t}^{i}))b_{t-1}}{P_{t}} + \frac{(1-\tau_{t}^{l})w_{t}l_{t}}{P_{t}} + \\
& \frac{(1-\tau_{t}^{k})r_{t}^{k}v_{t}k_{t-1}}{P_{t}} + \tau_{t}^{k}\delta_{t}^{\tau}k_{t-1}^{\tau} + \tau_{t}^{ic}i_{t} + \tau_{t}^{k}e_{t}^{\tau}i_{t} +  \frac{(1-\tau_{t}^{d})d_{t}}{P_{t}} + x_{t}
\end{split}
\end{equation}

\noindent\noindent Where all variables pre-determined at time $t$ have subscripts strictly less than time $t$.  $c$ represents consumption, $i$ investment, $b$ government bond holdings, $p$ price level, $l$ labor supply, $k$ capital stock, $x$ government transfers, $d$ dividends, $w$ the nominal wage rate, $v$ intensity of capital utilization. Taxes $\tau^{i}$, $\tau^{l}$, $\tau^{k}$, $\tau^{d}$ are taxes on interest income, labor income, capital income, and dividend income.  $r$ is the nominal interest rate on government bonds and $r^{k}$ is the rental rate on capital.  $k^{\tau}$ is the tax basis for the household's capital stock, $\delta^{\tau}$ is the rate of depreciation for tax purposes, $e^{\tau}$ is the rate of expensing for tax purposes, and $\tau^{ic}$ is the investment tax credit.


The law of motion of the household's capital stock is:
\begin{equation}
k_{t}=(1-\delta(v_{t}))k_{t-1} + i_{t}\left[1-S\left(\frac{i_{t}}{i_{t-1}}\right)\right]
\end{equation}

I'll assume that the investment adjustment cost function takes the form $S\left(\frac{i_{t}}{i_{t-1}}\right)=\frac{\gamma}{2}\left(\frac{i_{t}}{i_{t-1}}-1\right)^2$.  Such a functional form is common in the literature (e.g., \cn{SW2003}, \cn{CER2010}, \cn{Zubairy2010}, et al) and has the nice properties that in the steady state $S=0,$ $S'=0$ and $S''>0$, and that in a deterministic steady state, the adjustment costs are zero.

The law of motion of the household's tax basis for it's capital stock is:
\begin{equation}
k_{t}^{\tau}=(1-\delta_{t}^{\tau})k_{t-1}^{\tau} + i_{t}(1-e_{t}^{\tau})
\end{equation}


Assume also that $\delta(v_{t})$ takes the form: $\delta(v_{t})=\delta_{0}+\delta_{1}(v_{t}-1) + \frac{\delta_{2}}{2}(v_{t}-1)^2$.  This means that in the deterministic steady state, we can calibrate $\delta$ such that $v_{t}=1$, and $\delta(v)=\delta_{0}$ (see \cn{TY2010} on this point).


FOC's (to go in appendix):

\begin{equation}
\frac{\partial U}{\partial c_{t}}: u_{c}(c_{t},l_{t},g_{t})- \lambda_{t}(1+\tau_{t}^{c}) = 0
\end{equation}

\begin{equation}
\frac{\partial U}{\partial l_{t}}: u_{l}(c_{t},l_{t},g_{t})+u_{c}(c_{t},l_{t},g_{t})\frac{w_{t}(1-\tau_{t}^{l})}{P_{t}(1+\tau_{t}^{c})}  = 0
\end{equation}


 \begin{equation}
\begin{split}
\frac{\partial U}{\partial i_{t}}: & u_{c}(c_{t},l_{t},g_{t})\frac{(-1+\tau_{t}^{ic}+\tau_{t}^{k}e_{t}^{\tau})}{(1+\tau_{t}^{c})} + \\
& \beta E_{t}u_{c}(c_{t+1},l_{t+1},g_{t+1})\frac{1}{(1+\tau_{t+1}^{c})}\left[\frac{(1-\tau_{t+1}^{k})r_{t+1}^{k}v_{t+1}}{P_{t+1}}\left(1-S\left(\frac{i_{t}}{i_{t-1}}\right)-i_{t}\left(\frac{S'\frac{i_{t}}{i_{t-1}}}{i_{t-1}}\right)\right)+\tau_{t+1}^{k}\delta_{t+1}^{\tau}(1-e_{t}^{\tau})\right] + \\
&  \beta^{2}E_{t}\left[u_{c}(c_{t+2},l_{t+2},g_{t+2})\left(\frac{1-\tau_{t+2}^{k}}{1+\tau_{t+2}^{c}}\right)\frac{r_{t+2}^{k}v_{t+2}}{P_{t+2}}\gamma\left(\frac{i_{t+1}}{i_{t}}-1\right)\left(\frac{i_{t+1}}{i_{t}}\right)^{2}\right]= 0
  \end{split}
  \end{equation}


\begin{equation}
\begin{split}
\frac{\partial U}{\partial b_{t}}: & \frac{-u_{c}(c_{t},l_{t},g_{t})}{P_{t}(1+\tau_{t}^{c})}  +
\beta E_{t}u_{c}(c_{t+1},l_{t+1},g_{t+1})\left(\frac{(1+r_{t+1}(1-\tau_{t+1}^{i}))}{P_{t+1}(1+\tau_{t+1}^{c})}\right)  = 0
\end{split}
\end{equation}

\begin{equation}
\begin{split}
\frac{\partial U}{\partial v_{t}}: & u_{c}(c_{t},l_{t},g_{t})\left(\frac{(1-\tau_{t}^{k})r_{t}^{k}k_{t-1}}{P_{t}(1+\tau_{t}^{c})}\right) - \beta E_{t}u_{c}(c_{t+1},l_{t+1},g_{t+1})\left(\frac{(1-\tau_{t+1}^{k})r_{t+1}^{k}v_{t+1}\frac{\partial \delta(v_{t})}{\partial v_{t}}k_{t-1}}{P_{t+1}(1+\tau_{t+1}^{c})}\right)  = 0
\end{split}
\end{equation}


\subsubsection{Utility specification 0}

This utility function is similar to that in \cn{CER2010}, \cn{TY2010}, and \cn{Zubairy2010}.

\begin{equation}
u(c_{t},l_{t},g_{t})= \varepsilon_{t}^{b}\left(\frac{[c_{t}^{\zeta}(1-l_{t})^{(1-\zeta)}]^{1-\sigma}}{1-\sigma}+\chi^{g}\frac{g_{t}^{1-\sigma_{g}}}{1-\sigma_{g}}\right)
\end{equation}

in which the preference shock is assumed to follow the following AR(1) process:
\begin{equation}\label{EqPrefShock}
   \begin{split}
      \log(\ve_t^b) &= \rho_\ve \log(\ve_{t-1}^b) + \eta_{\ve,t} \\
      &\quad\text{where}\quad\eta_{\ve,t}\sim N(0,\sigma_\ve) \quad\text{and}\quad \rho_\ve\in[0,1)
   \end{split}
\end{equation}

FOC's with this utility function are:



\begin{align}
& \frac{\partial U}{\partial c_{t}}:\: \frac{\varepsilon_{t}^{b}\zeta c_{t}^{\zeta(1-\sigma)-1}(1-l_{t})^{(1-\zeta)(1-\sigma)}}{1+\tau_{t}^{c}} = \lambda_{t} \label{EqFOCc} \\
   & \frac{\partial U}{\partial l_{t}}:\: \epsilon^{b}*((1-\zeta)*(c^{\zeta(1-\sigma)}))*((1-l_{t})^{(1-\zeta)*(1-\sigma)-1})=\frac{(\lambda*w_{t}*(1-\tau_{t}^{l}))}{P_{t}(1+\tau_{t}^{c})}\label{EqFOCl} \\
      \begin{split}
      &\frac{\partial U}{\partial i_{t}}:\: \ve_{t}^b\left(\frac{c_{t}}{1-l_{t}}\right)^{(\zeta-1)(1-\sigma)}c_{t}^{-\sigma}\left(\frac{\tau_{t}^{ic}+\tau_{t}^{k}e_{t}^{\tau} - 1}{1+\tau_{t}^{c}}\right) + ... \\
      &\quad\quad\quad\quad \beta E_{t}\Biggl[\ve_{t+1}^b\left(\frac{c_{t+1}}{1-l_{t+1}}\right)^{(\zeta-1)(1-\sigma)}c_{t+1}^{-\sigma}... \\
      &\quad\quad\quad\quad \Biggl(\frac{[1-\tau_{t+1}^k]r_{t+1}^k v_{t+1}}{P_{t+1}(1+\tau_{t+1}^c)}\left[1-\frac{\gamma}{2}\left(\frac{i_{t}}{i_{t-1}}-1\right)^2 - \gamma\left(\frac{i_{t}}{i_{t-1}}\right)\left(\frac{i_{t}}{i_{t-1}}-1\right)\right] + \frac{\tau_{t+1}^k\delta_{t+1}^\tau(1-e_{t+1}^\tau)}{1+\tau_{t+1}^c}\Biggr)\Biggr] + ... \\
      &\quad\quad\quad\quad \beta^{2}E_{t}\Biggl[\ve_{t+2}^b\left(\frac{c_{t+2}}{1-l_{t+2}}\right)^{(\zeta-1)(1-\sigma)}c_{t+2}^{-\sigma}\frac{(1-\tau_{t+2}^k)r_{t+2}^k v_{t+2}}{P_{t+2}(1+\tau_{t+2}^c)}\gamma\left(\frac{i_{t+1}}{i_{t}}-1\right)\left(\frac{i_{t+1}}{i_{t}}\right)^{2}\Biggr]= 0
   \end{split} \label{EqFOCi} \\
   \begin{split}
      &\frac{\partial U}{\partial b_{t}}:\: \ve_{t}^b\left(\frac{c_{t}}{1-l_{t}}\right)^{(\zeta-1)(1-\sigma)}c_{t}^{-\sigma}\frac{1}{P_t(1+\tau_t^c)} = ... \\
      &\quad\quad\quad\quad \beta E_t\Biggl[\ve_{t+1}^b\left(\frac{c_{t+1}}{1-l_{t+1}}\right)^{(\zeta-1)(1-\sigma)}c_{t+1}^{-\sigma}\frac{1+r_{t+1}(1-\tau_{t+1}^i)}{P_{t+1}(1+\tau_{t+1}^c)}\Biggr]
   \end{split} \label{EqFOCb} \\
   \begin{split}
      &\frac{\partial U}{\partial v_{t}}:\: \ve_{t}^b\left(\frac{c_{t}}{1-l_{t}}\right)^{(\zeta-1)(1-\sigma)}c_{t}^{-\sigma}\frac{(1-\tau_t^k)r_t^k k_{t-1}}{P_{t}(1+\tau_t^c)} = ... \\
      &\quad\quad\quad\quad \beta E_t\Biggl[\ve_{t+1}^b\left(\frac{c_{t+1}}{1-l_{t+1}}\right)^{(\zeta-1)(1-\sigma)}c_{t+1}^{-\sigma}\frac{(1-\tau_{t+1}^k)r_{t+1}^k v_{t+1}}{P_{t+1}(1+\tau_{t+1}^c)}\bigl(\delta_1 + \delta_2[v_t - 1]\bigr)k_{t-1}\Biggr]
   \end{split}  \label{EqFOCv}\\
\end{align}


\subsubsection{Alternative Setup}

 \begin{equation}
\begin{split}
E_{0}\sum_{t=0}^{\infty} \beta^{t}\left[ u(c_{t},l_{t},g_{t})- \lambda_{t}
\left[(1+\tau_{t}^{c})c_{t}+\frac{b_{t}}{P_{t}}+i_{t}-\frac{(1+r_{t}(1-\tau_{t}^{i}))b_{t-1}}{P_{t}} - \frac{(1-\tau_{t}^{l})w_{t}l_{t}}{P_{t}} - \frac{(1-\tau_{t}^{k})r_{t}^{k}v_{t}k_{t-1}}{P_{t}} - \tau_{t}^{k}\delta_{t}^{\tau}k_{t-1}^{\tau} - \tau_{t}^{ic}i_{t} - \tau_{t}^{k}e_{t}^{\tau}i_{t} -  \frac{(1-\tau_{t}^{d})d_{t}}{P_{t}} - x_{t}\right]-Q_{t}
\left[k_{t}-(1-\delta(v_{t}))k_{t-1}-i_{t}\left(1-S\left[\frac{i_{t}}{i_{t-1}}\right]\right)\right]\right]
 \end{split}
  \end{equation}

FOCS:

\begin{equation}
\frac{\partial U}{\partial c_{t}}: u_{c}(c_{t},l_{t},g_{t})- \lambda_{t}(1+\tau_{t}^{c}) = 0
\end{equation}

\begin{equation}
\frac{\partial U}{\partial l_{t}}: u_{l}(c_{t},l_{t},g_{t})+\frac{\lambda_{t}w_{t}(1-\tau_{t}^{l})}{P_{t}} = 0
\end{equation}


 \begin{equation}
\begin{split}
\frac{\partial U}{\partial b_{t}}: \frac{\lambda_{t}}{P{t}} = \beta E_{t}\left[ \frac{\lambda_{t+1}(1+r_{t+1}(1-\tau_{t+1}^{i}))}{P_{t+1}} \right]
  \end{split}
  \end{equation}

\begin{equation}
\begin{split}
\frac{\partial U}{\partial v_{t}}: \frac{\lambda_{t}(1-\tau{t}^{k})r_{t}^{k}k_{t-1}}{P_{t}} = Q_{t}(\delta_{1}+\delta_{2}(v_{t}-1))k_{t-1}
\end{split}
\end{equation}

\begin{equation}
\begin{split}
\frac{\partial U}{\partial k_{t}}: Q_{t} = \beta E_{t} \left[\frac{\lambda_{t+1}(1-\tau_{t+1}^{k})r_{t+1}^{k}v_{t+1}}{P_{t+1}} + Q_{t+1}(1-\delta(v_{t+1}))\right]
\end{split}
\end{equation}

\begin{equation}
\begin{split}
\frac{\partial U}{\partial i_{t}}: \lambda_{t}(1-\tau_{t}^{ic}-\tau_{t}^{k}e_{t}^{\tau}) & = Q_{t}\left[1-S\left(\frac{i_{t}}{i_{t-1}}\right)-S'\left(\frac{i_{t}}{i_{t-1}}\right)\frac{i_{t}}{i_{t-1}} \right] + \\
&  \beta E_{t}\left[\lambda_{t+1}\tau_{t+1}^{k}\delta_{t+1}^{\tau}(1-e_{t}^{\tau}) + Q_{t+1}S'\left(\frac{i_{t+1}}{i_{t}}\right)\left(\frac{i_{t+1}}{i_{t}}\right)^{2}\right]
\end{split}
\end{equation}



\subsubsection{HH FOC's in the SS}

\begin{equation}
\frac{\partial U}{\partial c}:\frac{\zeta \bar{c}^{\zeta(1-\sigma)-1}(1-\bar{l})^{(1-\zeta)(1-\sigma)}}{(1+\tau^{c})} = \bar{\lambda}
\end{equation}

\begin{equation}
\frac{\partial U}{\partial l}: \frac{(1-\zeta)\bar{c}}{\zeta (1-\bar{l})} = \frac{\bar{w}(1-\tau^{l})}{(1+\tau^{c})}
\end{equation}


 \begin{equation}
\begin{split}
\frac{\partial U}{\partial i}:\left(1-\tau^{ic}-\tau^{k}e^{\tau}\right) = \beta \left[(1-\tau^{k})\bar{r}^{k}+ \tau^{k}\delta^{\tau}(1-e^{\tau})\right]
 \end{split}
  \end{equation}


\begin{equation}
\begin{split}
\frac{\partial U}{\partial b}: 1 = \beta(1+\bar{r}(1-\tau^{i}))
\end{split}
\end{equation}

\begin{equation}
\frac{\partial U}{\partial v}: \bar{k} = \beta \delta_{1}
\end{equation}


\subsubsection{SS of HH constraints/laws of motion}
\begin{itemize}
\item $\bar{k}=(1-\delta(\bar{v}))\bar{k}+\bar{i}\implies \bar{k}\delta(\bar{v})=\bar{i}$, if $\bar{v}=1$, then $\bar{i}=\bar{k}\delta_{0}$
\item $\delta(\bar{v}) = \delta_{0}+\delta_{1}(\bar{v}-1)+\frac{\delta_{2}}{2}(\bar{v}-1)^{2}$, if $\bar{v}=1$, then $\delta(\bar{v})=\delta_{0}$
\item $S(\bar{i}) = 0$
\item $\bar{k}^{\tau} = (1-\delta^{\tau})\bar{k}^{\tau}+\bar{i}(1-e^{\tau}) \implies \bar{k}^{\tau}\delta^{\tau}=\bar{i}(1-e^{\tau})$
\item Seems as though can normalize $\bar{P}=1$ and $\bar{v}=1$.  Problems with this??
\end{itemize}


\subsubsection{SS of non-HH equations}

\begin{itemize}
\item Resource constraint: $\bar{Y} = \bar{C}+\delta_{0}\bar{K}+\bar{G}$
\item Government Budget Constraint: $\tau^{c}\bar{C} + \tau^{l}\bar{L}\bar{w} - (\bar{r}(1-\tau^{i}))\bar{B} + \tau^{k}\bar{r}^{k}\bar{K} + \tau^{d}\bar{D} = \bar{G} + \bar{X} + \tau^{ic}\delta_{0}\bar{K} + \tau^{k}\delta_{0}\bar{K}(1-e^{\tau}) + \tau^{k}e^{\tau}\delta_{0}\bar{K} $  It might be that profits of the intermediate goods producers are zero in the SS- in which case, $\bar{D}=0$.
\item Technology shock: $\bar{z} = 1$
\item SS conditions from Intermediate Goods Producers:
	\begin{itemize}
	\item marginal cost: $ mc = \frac{\bar{w}(1-\alpha)\bar{r}^{k \alpha}}{\alpha^{\alpha}(1-\alpha)^{(1-\alpha)}}$
	\item labor-capital ratio: $\frac{\bar{K}}{\bar{L}}=\left(\frac{\alpha}{1-\alpha}\right)\frac{\bar{w}}{\bar{r}^{k}}$
	\item output: $\bar{Y} = \bar{K}^{\alpha}\bar{L}^{(1-\alpha)}$
	\end{itemize}
	\item Final Goods Producers: $\bar{P}=1$ (\emph{I think}).
\end{itemize}



%\subsubsection{Log-linearizing HH FOCs and other necessary conditions:}
%
%Let $\hat{X}_{t}=\left(\frac{X_{t}-\bar{X}}{\bar{X}}\right)$.  This is the percentage deviation from the steady state times the steady state value.
%
%\emph{These are not right.  I need to update them, though it's not that important if using Dynare...}
%
%\begin{equation}
%\frac{\partial U}{\partial c_{t}}: (\zeta(1-\sigma)-1)\hat{c}_{t} - \frac{\bar{l}(1-\zeta)(1-\sigma)}{(1-\bar{l})}\hat{l}_{t} - \hat{\lambda}_{t}
% - \frac{\bar{\lambda}\tau^{c}\hat{\tau}_{t}^{c}}{\zeta\bar{c}^{\zeta(1-\sigma)-1}(1-\bar{l})^{(1-\zeta)(1-\sigma)}} = 0
%\end{equation}
%
%\begin{equation}
%\frac{\partial U}{\partial l_{t}}: \frac{(1-\zeta)\bar{c}}{\zeta \bar{l}}(\hat{l}_{t}-\hat{c}_{t}) + \frac{(1+\bar{l})(1-\tau^{l})\bar{w}}{1+\tau^{c}}\left[\frac{\hat{l}_{t}}{1+\bar{l}} + \hat{w}_{t} - \frac{\tau^{l}\hat{\tau}_{t}^{l}}{(1-\tau^{l})} - \hat{P}_{t} - \frac{\tau^{c}\hat{\tau}_{t}^{c}}{1+\tau^{c}}\right] = 0
%\end{equation}
%
%
% \begin{equation}
%\begin{split}
%\frac{\partial U}{\partial i_{t}}:  & (-1+\tau^{ic}+e^{\tau})\left[\hat{c}_{t}-\frac{\bar{l}\hat{l}_{t}}{(1-\bar{l})}+\hat{\varepsilon}_{t}^{b}-\frac{\tau^{c}\hat{\tau}_{t}^{c}}{1+\tau^{c}}\right] +
% \tau^{ic}\hat{\tau}_{t}^{ic} + e^{\tau}\hat{e}_{t}^{\tau} + \\
% & \beta E_{t}\left[((1-\tau^{k})\bar{r}^{k}+\tau^{k}\delta^{\tau}(1-e^{\tau}))\left(\hat{c}_{t+1}-\frac{\bar{l}\hat{l}_{t+1}}{(1-\bar{l})}+\hat{\varepsilon}_{t+1}^{b}-\frac{\tau^{c}\hat{\tau}_{t+1}^{c}}{1+\tau^{c}}\right) \right] \\
%  & + \beta E_{t}\left[ (1-\tau^{k})\bar{r}^{k}\hat{r}_{t+1}^{k} - (\bar{r}^{k}\delta^{\tau}(1-e^{\tau}))\tau^{k}\hat{\tau}_{t+1}^{k} - \tau^{k}\delta^{\tau}e^{\tau}\hat{e}_{t}^{\tau} - (1-\tau^{k})\bar{r}^{k}\gamma\hat{i}_{t} +
%   (1-\tau^{k})\bar{r}^{k}\gamma\hat{i}_{t-1} \right] \\
%   & = 0
% \end{split}
%  \end{equation}
%
%
%\begin{equation}
%\begin{split}
%\frac{\partial U}{\partial b_{t}}: & (\zeta(1-\sigma)-1)\hat{c}_{t} - \frac{(1-\zeta)(1-\sigma)\bar{l}\hat{l}_{t}}{1-\bar{l}} - \hat{P}_{t} + \hat{\varepsilon}_{t}^{b} - \frac{\tau^{c}\hat{\tau}_{t}^{c}}{1+\tau^{c}} - \\
%& \beta E_{t}\left[(1+\bar{r}(1-\tau^{i}))\left[(\zeta(1-\sigma)-1)\hat{c}_{t+1} - \frac{(1-\zeta)(1-\sigma)\bar{l}\hat{l}_{t+1}}{1-\bar{l}} - \hat{P}_{t+1} + \hat{\varepsilon}_{t+1}^{b} - \frac{\tau^{c}\hat{\tau}_{t+1}^{c}}{1+\tau^{c}}\right]
% + (1-\tau_{i})\bar{r}\hat{r}_{t+1} \right] \\
% & = 0
%\end{split}
%\end{equation}
%
%\begin{equation}
%\begin{split}
%\frac{\partial U}{\partial v_{t}}: & \bar{k}\left[(\zeta(1-\sigma)-1)\hat{c}_{t} - \frac{(1-\sigma)(1-\zeta)\bar{l}\hat{l}_{t}}{(1-\bar{l})} - \frac{\tau^{k}\hat{\tau}_{t}^{k}}{(1-\tau^{k})} + \hat{r}_{t}^{k} + \hat{k}_{t-1} - \frac{\tau^{c}\hat{\tau}_{t}^{c}}{1+\tau^{c}} + \hat{\varepsilon}_{t}^{b}\right] \\
%& - \beta E_{t} \left[(\zeta(1-\sigma)-1)\hat{c}_{t+1} - \frac{(1-\sigma)(1-\zeta)\bar{l}\hat{l}_{t+1}}{(1-\bar{l})} - \frac{\tau^{k}\hat{\tau}_{t+1}^{k}}{(1-\tau^{k})} + \hat{r}_{t+1}^{k} + \hat{v}_{t+1} + \frac{\delta_{2}\hat{v}_{t}}{\delta_{1}} - \frac{\tau^{c}\hat{\tau}_{t+1}^{c}}{1+\tau^{c}} + \hat{\varepsilon}_{t+1}^{b}\  \right] = 0
%\end{split}
%\end{equation}
%
%\begin{equation}
%\text{Resource Constraint}: \bar{Y}(\hat{Y}_{t} - \alpha\hat{K}_{t}-(1-\alpha)\hat{L}_{t}) = 0
%\end{equation}
%
%\begin{equation}
%\text{Aggregate Tech Shock}: \hat{z}_{t} = \rho\hat{z}_{t-1} + \epsilon_{t}
%\end{equation}
%
%\begin{equation}
%\text{Law of Motion of Capital Stock}: \bar{K}\hat{K}_{t}-(1-\delta_{0})\bar{K}\hat{K}_{t-1} + \delta_{1}\bar{K}\hat{V}_{t}  - \bar{I}\hat{I}_{t} = 0
%\end{equation}
%
%\begin{equation}
%\text{Law of Motion of Tax Basis of Capital Stock}: \bar{K}^{\tau}(\hat{K}_{t}^{\tau}-(1-\delta^{\tau})\hat{K}_{t-1}^{\tau}+\delta^{\tau}\hat{\delta}_{t}^{\tau}) -
%\bar{I}((1-e^{\tau})\hat{I}_{t}-e^{\tau}\hat{e}_{t}^{\tau}) = 0
%\end{equation}
%
%\begin{equation}
%\begin{split}
%\text{Gov't Budget Constraint}:  & \bar{C}\tau^{c}(\hat{\tau}_{t}^{c}+\hat{C}_{t}) + \tau^{l}\bar{L}\bar{w}(\hat{\tau}_{t}^{l}+\hat{L}_{t}+\hat{w}_{t}-\hat{P}_{t}) - \bar{B}(1+\bar{r}(1-\tau^{i}))(\hat{B}_{t}-\hat{P}_{t}) \\
%& - \bar{B}(1-\tau^{i})\bar{r}\hat{r}_{t} + \bar{B}\bar{r}\tau^{i}\hat{\tau}_{t}^{i} + \tau^{k}\bar{r}^{k}\bar{K}(\hat{\tau}_{t}^{k}+\hat{r}_{t}^{k}+\hat{V}_{t}+\hat{K}_{t-1}) -  \\
%&  \tau^{k}\delta^{\tau}\bar{K}^{\tau}(\hat{\tau}_{t}^{k}+\hat{\delta}_{t}^{\tau}+\hat{K}_{t-1}) - \tau^{ic}\bar{I}(\hat{\tau}_{t}^{ic}+\hat{I}_{t}) - e^{\tau} \bar{I}(\hat{e}_{t}^{\tau}+\hat{I}_{t}) + \\
% &  \bar{B}(\hat{B}_{t}-\hat{P}_{t}) - \bar{X}\hat{X}_{t} - \bar{G}\hat{G}_{t} = 0
%\end{split}
%\end{equation}


\subsubsection{Steady state values of state variables:}

From the Calvo pricing equation, we can normalize $\bar{P}$:
\begin{equation}
\bar{P} = 1
\end{equation}

We can calibrate $\delta_{0}$ such that we can normalize our SS capital utilization:
\begin{equation}
\bar{V} = 1
\end{equation}

We can normalize the mean of the TFP process (and know that the disturbance term = 0 in the SS):
\begin{equation}
\bar{Z} = 1
\end{equation}

We can normalize the mean of the taste shock (and know that the disturbance term = 0 in the SS):
\begin{equation}
\bar{\varepsilon}^{b} = 1
\end{equation}

We know that the disturbance term to the price markup is zero in the SS, thus:
\begin{equation}
\bar{\lambda}_{p} = \lambda_{p}
\end{equation}

We can solve for the SS interest rate directly from the HH FOC for demand for gov't bonds (or from the Taylor Rule):
\begin{equation}
\bar{r} = \frac{1-\beta}{\beta(1-\tau^{i})}
\end{equation}

The FOC for the HH investment decision, we can find the SS return on capital:
\begin{equation}
\bar{r}^{k} = \frac{\left(\frac{1-\tau^{ic}-\tau^{k}e^{\tau}}{\beta}\right)-\tau^{k}\delta^{\tau}(1-e^{\tau})}{(1-\tau^{k})}
\end{equation}

The FOC for labor demand from the int goods producer's problem gives the SS wage rate:
\begin{equation}
\bar{w} = (1-\alpha)\left(\frac{\bar{r}^{k}}{\alpha}\right)^{\frac{-\alpha}{1-\alpha}}
\end{equation}

Calibration gives the SS labor supply (I think this implies the Frisch elasticity of labor paramter = 1 -- see \cn{Zubairy2010}):
\begin{equation}
\bar{L} = 0.5
\end{equation}

The FOC for effective capital demand from the int goods producer's problem gives the SS level of capital:
\begin{equation}
\bar{K} = \left(\frac{\alpha}{1-\alpha}\right)\frac{\bar{w}\bar{L}}{\bar{r}^{k}}
\end{equation}

The law of motion for the tax basis of the capital stock gives the SS value for the tax basis of the capital stock:
\begin{equation}
\bar{K}^{\tau} = \frac{\delta_{0}\bar{K}(1-e^{\tau})}{\delta^{\tau}}
\end{equation}

The law of motion for the capital stock gives SS investment:
\begin{equation}
\bar{I} = \delta_{0}\bar{K}
\end{equation}

The int goods producers' production function gives the aggregate level of output in the SS:
\begin{equation}
\bar{Y} = \bar{K}^{\alpha}\bar{L}^{(1-\alpha)}
\end{equation}

The int goods producers' profit function gives aggregate profits in the SS:
\begin{equation}
\bar{D} = \bar{Y} - \bar{w}\bar{L} - \bar{r}^{k}\bar{K}
\end{equation}


The HH FOC for labor gives SS consumption:
\begin{equation}
\bar{C} = \frac{\bar{w}(1-\tau^{l})(1-\bar{L})\zeta}{(1-\zeta)(1+\tau^{c})}
\end{equation}

The HH FOC for consumption gives SS marginal utility of consumption (the shadow price of consumption):
\begin{equation}
\bar{\lambda} = \frac{\zeta \bar{c}^{\zeta(1-\sigma)-1}(1-\bar{l})^{(1-\zeta)(1-\sigma)}}{(1+\tau^{c})}
\end{equation}

The resource constraint gives SS government spending:
\begin{equation}
\bar{G} = \bar{Y}-\bar{C}-\bar{I}
\end{equation}

The gov't budget constraint gives SS bond issuance necessary to finance $\bar{G}$:
\begin{equation}
\bar{B} = \frac{\left[\tau^{c}\bar{C} + \tau^{l}\bar{L}\bar{w} + \tau^{k}\bar{r}^{k}\bar{K} + \tau^{d}\bar{D} - \tau^{ic}\bar{I} - \tau^{k}\delta^{\tau}\bar{K}^{\tau} - \tau^{k}e^{\tau}\bar{I}-(1-\gamma_{x})\bar{G}\right]}{\bar{r}(1-\tau^{i})}
\end{equation}

The assumption regarding what fraction of the federal budget goes to X gives SS government transfers:
\begin{equation}
\bar{X} = \gamma_{x}*\Biggl[\tau^{c}\bar{C} + \tau^{l}\bar{L}\bar{w} + \bar{B} + \tau^{k}\bar{r}^{k}\bar{K} + \tau^{d}\bar{D} - \tau^{ic}\bar{I} - \tau^{k}\delta^{\tau}\bar{K}^{\tau} - \tau^{k}e^{\tau}\bar{I} - \bar{B}\left[1+\bar{r}(1-\tau^{i})\right]\Biggr]
\end{equation}









\subsection{Firms}

	\subsubsection{Final Goods Producers}

Final goods producers aggregate differentiated inputs to produce a homogenous output.  The Dixit-Stiglitz aggregation function is given by;

\begin{equation}
\label{aggY}
Y_{t}=\left(\int_{0}^{1} y_{i,t}^{\frac{1}{1+\lambda_{p,t}}}di\right)^{1+\lambda_{p,t}},
\end{equation}

\noindent\noindent where $log(\lambda_{p,t})=\rho_{\lambda}log(\lambda_{p,t-1})+(1-\rho_{\lambda})log(\lambda_{p})+\eta_{t}^{p}$ is the stochastic price markup in the intermediate goods market.  We assume that $\eta_{t}^{p}\sim N(0,\sigma_\lambda)$

Maximizing profits (or minimizing costs) subject to Equation \ref{aggY} results in demand for intermediate input $y_{i,t}$ of:

\begin{equation}
y_{i,t}=\left(\frac{p_{i,t}}{P_{t}}\right)^{-\frac{1+\lambda_{p,t}}{\lambda_{p,t}}}Y_{t} ,
\end{equation}

\noindent\noindent where the price of a unit of the final, homogenous output, $P_{t}$, is given by:

\begin{equation}
P_{t}=\left(\int_{0}^{1}p_{i,t}^{-\frac{1}{\lambda_{p,t}}}di\right)^{-\lambda_{p,t}}
\end{equation}

\subsubsection{Working Through the Final Goods Producer's problem}

Objective function:

\begin{equation}
\max_{y_{i,t}} P_{t}Y_{t}^{d} - \int_{0}^{1}p_{i,t}y_{i,t}di
\end{equation}

The FOC's are:

\begin{equation}
p_{t}(1+\lambda_{p,t})\left(\int_{0}^{1}y_{i,t}^{\left(\frac{1}{1+\lambda_{p,t}}\right)}di\right)^{\lambda_{p,t}}\left(\frac{1}{1+\lambda_{p,t}}\right)y_{i,t}^{\frac{1}{1+\lambda_{p,t}}-1}-p_{i,t} = 0, \forall i
\end{equation}

Dividing the FOCs for intermediate inputs $i$ and $j$ $\implies$
\begin{equation}
\frac{p_{i,t}}{p_{j,t}} = \left(\frac{y_{i,t}}{y_{j,t}}\right)^{\frac{1}{1+\lambda_{p,t}}-1} = \left(\frac{y_{i,t}}{y_{j,t}}\right)^{\frac{-\lambda_{p,t}}{1+\lambda_{p,t}}}
\end{equation}

Rearranging:
\begin{equation}
\begin{split}
&\implies p_{i,t} = \left(\frac{y_{j,t}}{y_{i,t}}\right)^{\frac{\lambda_{p,t}}{1+\lambda_{p,t}}}p_{j,t} \\
& \implies p_{i,t} = y_{j,t}^{\frac{\lambda_{p,t}}{1+\lambda_{p,t}}}p_{j,t}y_{i,t}^{\frac{-\lambda_{p,t}}{1+\lambda_{p,t}}} \\
& \implies p_{i,t}y_{i,t} = p_{j,t}y_{j,t}^{\left(\frac{\lambda_{p,t}}{1+\lambda_{p,t}}\right)}y_{i,t}^{\frac{1}{1+\lambda_{p,t}}}
\end{split}
\end{equation}

Then integrate this condition to yield:
\begin{equation}
\begin{split}
\int_{0}^{1}p_{i,t}y_{i,t}di & = p_{j,t}y_{j,t}^{\left(\frac{\lambda_{p,t}}{1+\lambda_{p,t}}\right)}\int_{0}^{1}y_{i,t}^{\frac{1}{1+\lambda_{p,t}}}di  \\
& = p_{j,t}y_{j,t}^{\left(\frac{\lambda_{p,t}}{1+\lambda_{p,t}}\right)}(Y_{t})^{\frac{1}{1+\lambda_{p,t}}}
\end{split}
\end{equation}

The zero profit condition implies $P_{t}Y_{t}=\int_{0}^{1}p_{i,t}y_{i,t}di$.  Plugging this into the above and we find:
\begin{equation}
\begin{split}
P_{t}Y_{t} &= p_{j,t}y_{j,t}^{\left(\frac{\lambda_{p,t}}{1+\lambda_{p,t}}\right)}(Y_{t})^{\frac{1}{1+\lambda_{p,t}}} \\
\implies &P_{t}Y_{t}=p_{j,t}y_{j,t}^{\frac{\lambda_{p,t}}{1+\lambda_{p,t}}}(Y_{t})^{\frac{1}{1+\lambda_{p,t}}} \\
\implies & P_{t} =  p_{j,t}y_{j,t}^{\frac{\lambda_{p,t}}{1+\lambda_{p,t}}}(Y_{t})^{\frac{-\lambda_{p,t}}{1+\lambda_{p,t}}}
\end{split}
\end{equation}

Which implies the demand function (just rearranging terms):
\begin{equation}
\begin{split}
y_{i,t} =  \left(\frac{p_{i,t}}{P_{t}}\right)^{\frac{-(1+\lambda_{p,t})}{\lambda_{p,t}}}Y_{t}, \forall i
\end{split}
\end{equation}

To find $P_{t}$ use the zero profit condition:
\begin{equation}
\begin{split}
& P_{t}Y_{t} = \int_{0}^{1}p_{i,t}y_{i,t}^{\left(\frac{\lambda_{p,t}}{1+\lambda_{p,t}}\right)}(Y_{t})^{\frac{1}{1+\lambda_{p,t}}}di \\
\implies & P_{t}Y_{t} = \int_{0}^{1}p_{i,t}\left(\left(\frac{p_{i,t}}{P_{t}}\right)^{\frac{-(1+\lambda_{p,t})}{\lambda_{p,t}}}Y_{t}\right)^{\left(\frac{\lambda_{p,t}}{1+\lambda_{p,t}}\right)}(Y_{t})^{\frac{1}{1+\lambda_{p,t}}}di \\
\implies & P_{t}Y_{t} = \int_{0}^{1}p_{i,t}\left(\frac{p_{i,t}}{P_{t}}\right)^{\frac{-(1+\lambda_{p,t})}{\lambda_{p,t}}}Y_{t}di \\
\implies & P_{t}Y_{t} = \int_{0}^{1}p_{i,t}p_{i,t}^{\frac{-(1+\lambda_{p,t})}{\lambda_{p,t}}}P_{t}^{\frac{(1+\lambda_{p,t})}{\lambda_{p,t}}}Y_{t}di \\
\implies & P_{t}Y_{t} = \int_{0}^{1}p_{i,t}^{\frac{-1}{\lambda_{p,t}}}di \:  Y_{t}P_{t}^{\frac{1+\lambda_{p,t}}{\lambda_{p,t}}} \\
\implies & P_{t} = \int_{0}^{1}p_{i,t}^{\frac{-1}{\lambda_{p,t}}}di \: P_{t}^{\frac{1+\lambda_{p,t}}{\lambda_{p,t}}} \\
\implies & P_{t}^{\frac{-1}{\lambda_{p,t}}} = \int_{0}^{1}p_{i,t}^{\frac{-1}{\lambda_{p,t}}}di \\
\implies & P_{t} = \left(\int_{0}^{1}p_{i,t}^{\frac{-1}{\lambda_{p,t}}}di\right)^{-\lambda_{p,t}} \\
\end{split}
\end{equation}



\subsubsection{Intermediate Goods Producers}

Each intermediate-goods producing firm has a monopoly on it's heterogeneous output.  Following \cn{Calvo1983}, firms can reset prices each period with probability $(1-\theta)$.  Firm's that cannot reset prices must rent labor and capital inputs to meet demand at last period's price, $p_{i,t-1}$.  Each intermediate good's producer faces the same production function:

\begin{equation}
\label{int_prodfunc}
y_{i,t}=z_{t}\tilde{k}_{i,t}^{\alpha}l_{i,t}^{1-\alpha},
\end{equation}

\noindent\noindent where $\tilde{k}_{i,t}=v_{t}k_{i,t-1}$ are the effective units of capital rented from the households, $l_{i,t}$ is the effective labor rented from households, $z_{t}$ is a serially correlated shock to firm productivity (which affect all intermediate-goods producers equally).

We assume $z_{t}$ follows the following AR(1) process:
\begin{equation}\label{EqProdShock}
   \log(z_{t}) = \rho_z\log(z_{t-1}) + \eta_{z,t} \quad\text{where}\quad\eta_{z,t}\sim N(0,\sigma_z), \quad\text{and}\quad \rho_z\in[0,1)
\end{equation}

There is no entry or exit.

The problem of intermediate-goods producers who can optimally reset their prices can be solved in two stages.  In the first, the demand for factor inputs is found, taking the rental rate for capital, $r_{t}^{k}$, and the wage rate, $w_{t}$, as given.  This stage yields the relative factor demands and the marginal cost of production.  In the second stage, firm's take their marginal cost function as given, and choose the their, $p_{i,t}$ to maximize their expected, discounted profits.

The firms' per-period profit function takes the following form:
\begin{equation}
   d_{i,t} = p_{i,t}y_{i,t} - r_t^k\tilde{k}_{i,t} - w_t l_{i,t} \quad\forall i,t
\end{equation}
A firm's objective function when it chooses price is to maximize expected profits in the case that the current price cannot be changed.

\begin{equation}
V=\max_{p_{i,t},\tilde{k}_{i,t},l_{i,t}}E_{t}\Biggl(\sum_{j=0}^{\infty}\beta^{t+j}\theta^{j}\frac{\lambda_{t+j}}{\lambda_{t}}\left[p_{i,t}y_{i,t+j}-r_{t+j}^{k}\tilde{k}_{i,t+j}-w_{t+j}l_{i,t+j}\right]\Biggr)
\end{equation}

FOC's:

\begin{equation}
\label{foc_int_k}
\frac{\partial V}{\partial \tilde{k}_{i,t}}:\frac{\alpha p_{i,t}y_{i,t}}{\tilde{k}_{i,t}}-r_{t}^{k}=0
\end{equation}

\begin{equation}
\label{foc_int_l}
\frac{\partial V}{\partial l_{i,t}}:\frac{(1-\alpha) p_{i,t}y_{i,t}}{l_{i,t}}-w_{t}=0
\end{equation}

Dividing Equation \ref{foc_int_k} by Equation \ref{foc_int_l} and rearranging some terms, one gets:

\begin{equation}
\label{k_l_ratio}
\frac{\tilde{k}_{i,t}}{l_{i,t}}=\left(\frac{\alpha}{1-\alpha}\right)\frac{w_{t}}{r_{t}^{k}}
\end{equation}


Plugging this ratio into Equation \ref{int_prodfunc} and setting it equal to 1, one can find the amount of labor needed to produce one unit of output in terms of $w_{t}, r{t}^{k}, z_{t}$, and the parameter $\alpha$.  Rearranging \ref{k_l_ratio} to put $\tilde{k}_{i,t}$ in terms of $l_{i,t}$, factor input prices, and parameters then writing the cost function ($r_{t+j}^{k}\tilde{k}_{i,t+j}+w_{t+j}l_{i,t+j}$) in terms of $l_{i,t}$, factor input prices, and parameters, one can substitute in the equation for $l_{i,t}$ This yields the marginal cost of production in real terms (this procedure is described in \cn{FVRR2006}):

\begin{equation}
mc_{t}=\frac{w_{t}^{(1-\alpha)}(r_{t}^{k})^{\alpha}}{p_{t}z_{t}\alpha^{\alpha}(1-\alpha)^{(1-\alpha)}}
\end{equation}

\noindent\noindent N.B., the marginal cost function, $mc_{t}$ does not depend upon $i$.  This is because all firms receive the same technology shock and face the same factor input prices.

The second stage, were firms optimally choose price, can be setup as a profit maximization problem, taking $mc_{t}$ as given.  Assuming an interior solution, the problem can be written as:

\begin{equation}
V=\max_{p_{i,t}}E_{t}\Biggl(\sum_{j=0}^{\infty}\beta^{j}\theta^{j}\frac{\lambda_{t+j}}{\lambda_{t}}\left[\left(\frac{p_{i,t}}{P_{t+j}}-mc_{t+j}\right)y_{i,t+j}\right]\Biggr),
\end{equation}

\noindent\noindent subject to:
\begin{equation}
y_{i,t+j}=\left(\frac{p_{i,t}}{P_{t+j}}\right)^{-\left(\frac{1+\lambda_{p,t+j}}{\lambda_{p,t+j}}\right)}Y_{t+j}
\end{equation}

\noindent\noindent where $Y_{t+j}$ is the demand for final goods in period $t+j$.  N.B. there is no $p_{i,t+j}$ because prices are fixed at their value chose in period $t$.


Rewriting the second-stage problem gives:

\begin{equation}
\begin{split}
 & V=\max_{p_{i,t}} E_{t}\left(\sum_{j=0}^{\infty}\beta^{j}\theta^{j}\frac{\lambda_{t+j}}{\lambda_{t}}\left[\left(\frac{p_{i,t}}{P_{t+j}}-mc_{t+j}\right)y_{i,t+j}\right]\right)
\\
\implies & V=\max_{p_{i,t}} E_{t}\left(\sum_{j=0}^{\infty} \beta^{j}\theta^{j}\frac{\lambda_{t+j}}{\lambda_{t}}\left[\left(\frac{p_{i,t}}{P_{t+j}}-mc_{t+j}\right)\left(\frac{p_{i,t}}{P_{t+j}}\right)^{-\frac{1+\lambda_{p,t+j}}{\lambda_{p,t+j}}}Y_{t+j}\right]\right) \\
 \implies & V=\max_{p_{i,t}} E_{t}\left(\sum_{j=0}^{\infty} \beta^{j}\theta^{j}\frac{\lambda_{t+j}}{\lambda_{t}}\left[\left(\frac{p_{i,t}}{P_{t+j}}\right)^{\frac{-1}{\lambda_{p,t+j}}}-\left(\frac{p_{i,t}}{P_{t+j}}\right)^{-\frac{1+\lambda_{p,t+j}}{\lambda_{p,t+j}}}mc_{t+j}\right]Y_{t+j}\right) \\
 \implies & V=\max_{p_{i,t}} E_{t}\left(\sum_{j=0}^{\infty} \beta^{j}\theta^{j}\frac{\lambda_{t+j}}{\lambda_{t}}\left[\left(\prod_{s=1}^{j}\frac{1}{\pi_{t+s}}\frac{p_{i,t}}{P_{t}}\right)^{\frac{-1}{\lambda_{p,t+j}}}-\left(\prod_{s=1}^{j}\frac{1}{\pi_{t+s}}\frac{p_{i,t}}{P_{t}}\right)^{-\frac{1+\lambda_{p,t+j}}{\lambda_{p,t+j}}}mc_{t+j}\right]Y_{t+j}\right)
\end{split}
\end{equation}

In the above, $\pi_{t}=\frac{P_{t}}{P_{t-1}}$.

The FOC for this problem (where the intermediate-goods producer is choosing $p_{i,t}$) is given by:

\begin{equation}
\begin{split}
 & \frac{\partial V}{\partial p_{i,t}}: E_{t}\left(\sum_{j=0}^{\infty} \beta^{j}\theta^{j}\frac{\lambda_{t+j}}{\lambda_{t}}\left[\frac{-1}{\lambda_{p,t+j}}\left(\prod_{s=1}^{j}\frac{1}{\pi_{t+s}}\frac{p_{i,t}}{P_{t}}\right)^{\frac{-1}{\lambda_{p,t+j}}}\left(\frac{1}{p_{i,t}}\right)+\left(\frac{1+\lambda_{p,t+j}}{\lambda_{p,t+j}}\right)\left(\prod_{s=1}^{j}\frac{1}{\pi_{t+s}}\frac{p_{i,t}}{P_{t}}\right)^{-\frac{1+\lambda_{p,t+j}}{\lambda_{p,t+j}}}\frac{mc_{t+j}}{p_{i,t}}\right]Y_{t+j}\right)=0 \\
\implies & \frac{\partial V}{\partial p_{i,t}}: E_{t}\left(\sum_{j=0}^{\infty} \beta^{j}\theta^{j}\frac{\lambda_{t+j}}{\lambda_{t}}\left(\prod_{s=1}^{j}\frac{1}{\pi_{t+s}}\frac{1}{p_{i,t}}\right)^{-\frac{1+\lambda_{p,t+j}}{\lambda_{p,t+j}}}\left[\frac{-1}{\lambda_{p,t+j}}\left(\prod_{s=1}^{j}\frac{1}{\pi_{t+s}}\frac{p_{i,t}}{P_{t}}\right)+\left(\frac{1+\lambda_{p,t+j}}{\lambda_{p,t+j}}\right)mc_{t+j}\right]Y_{t+j}\right)=0
\end{split}
\end{equation}

Rearranging:
\begin{equation}
\begin{split}
\implies & \frac{\partial V}{\partial p_{i,t}}: E_{t}  \left(\sum_{j=0}^{\infty} \beta^{j}\theta^{j}\frac{\lambda_{t+j}}{\lambda_{t}}\left(\prod_{s=1}^{j}\frac{1}{\pi_{t+s}}\frac{1}{p_{i,t}}\right)^{-\frac{1+\lambda_{p,t+j}}{\lambda_{p,t+j}}}\left[\frac{1}{\lambda_{p,t+j}}\left(\prod_{s=1}^{j}\frac{1}{\pi_{t+s}}\frac{p_{i,t}}{P_{t}}\right)\right]Y_{t+j}\right) \\
& = E_{t}  \left(\sum_{j=0}^{\infty} \beta^{j}\theta^{j}\frac{\lambda_{t+j}}{\lambda_{t}}\left(\prod_{s=1}^{j}\frac{1}{\pi_{t+s}}\frac{1}{p_{i,t}}\right)^{-\frac{1+\lambda_{p,t+j}}{\lambda_{p,t+j}}}\left[\left(\frac{1+\lambda_{p,t+j}}{\lambda_{p,t+j}}\right)mc_{t+j}\right]Y_{t+j}\right)
\end{split}
\end{equation}


Or (using notation without gross inflation):
\begin{equation}
\begin{split}
\implies  &  \frac{\partial V}{\partial p_{i,t}}:\: E_{t}\Biggl(\sum_{j=0}^{\infty}\beta^{j}\theta^{j}\left(\frac{p_{i,t}}{P_{t+j}}\right)^{-\frac{1+\lambda_{p,t+j}}{\lambda_{p,t+j}}}\frac{Y_{t+j}\lambda_{t+j}}{\lambda_{p,t+j}\lambda_{t}}\left[\frac{(1+\lambda_{p,t+j})mc_{t+j}P_{t+j} - p_{i,t}}{p_{i,t}P_{t+j}}\right]\Biggr) = 0 \ \ \forall i,t
\end{split}
\end{equation}

Which implies:
\begin{equation}
\begin{split}
 &   E_{t}\Biggl(\sum_{j=0}^{\infty}\beta^{j}\theta^{j}\left(\frac{p_{i,t}}{P_{t+j}}\right)^{-\frac{1+\lambda_{p,t+j}}{\lambda_{p,t+j}}}\frac{Y_{t+j}\lambda_{t+j}}{\lambda_{p,t+j}\lambda_{t}}\left[\frac{(1+\lambda_{p,t+j})mc_{t+j}P_{t+j}}{p_{i,t}P_{t+j}}\right]\Biggr) =  \\
 & E_{t}\Biggl(\sum_{j=0}^{\infty}\beta^{j}\theta^{j}\left(\frac{p_{i,t}}{P_{t+j}}\right)^{-\frac{1+\lambda_{p,t+j}}{\lambda_{p,t+j}}}\frac{Y_{t+j}\lambda_{t+j}}{\lambda_{p,t+j}\lambda_{t}}\left[\frac{p_{i,t}}{p_{i,t}P_{t+j}}\right]\Biggr)
\end{split}
\end{equation}

Which can be written as:
\begin{equation}
\begin{split}
 &   p_{i,t}^{*}=\frac{E_{t}\Biggl(\sum_{j=0}^{\infty}\beta^{j}\theta^{j}\left(\frac{p_{i,t}}{P_{t+j}}\right)^{-\frac{1+\lambda_{p,t+j}}{\lambda_{p,t+j}}}\frac{Y_{t+j}\lambda_{t+j}}{\lambda_{p,t+j}}\left[(1+\lambda_{p,t+j})mc_{t+j}\right]\Biggr)}{E_{t}\Biggl(\sum_{j=0}^{\infty}\beta^{j}\theta^{j}\left(\frac{p_{i,t}}{P_{t+j}}\right)^{-\frac{1+\lambda_{p,t+j}}{\lambda_{p,t+j}}}\frac{Y_{t+j}\lambda_{t+j}}{\lambda_{p,t+j}P_{t+j}}\Biggr)}
\end{split}
\end{equation}

Assume a symmetric equilibrium ($\implies p_{i,t}^{*}=p_{t}^{*}, \ \ \forall i$).

Calvo pricing (with zero indexation) implies $P_{t}^{\frac{-1}{\lambda_{p,t}}}= \theta P_{t-1}^{\frac{-1}{\lambda_{p,t}}} +(1-\theta)p_{t}^{*\frac{-1}{\lambda_{p,t}}}$ (you can get this from the definition of the Calvo model and the aggregate price index from the final goods producers problem).  One can divide this equation through by $P_{t}^{\frac{-1}{\lambda_{p,t}}}$ and get $1=\theta \left(\frac{P_{t-1}}{P_{t}}\right)^{\frac{-1}{\lambda_{p,t}}}+(1-\theta)\left(\frac{p_{t}^{*}}{P_{t}}\right)^{\frac{-1}{\lambda_{p,t}}}$.  Using the gross inflation variable, $\pi_{t}$ defined above, we can write: $\left(\frac{p_{t}^{*}}{P_{t}}\right)=\left(\frac{1-\theta\pi_{t}^{\frac{1}{\lambda{p,t}}}}{1-\theta} \right)^{-\lambda_{p,t}}$.

Using this, we can write the equation for $p_{t}^{*}$ as:

Which can be written as:
\begin{equation}
\begin{split}
 &   p_{t}^{*}=\frac{E_{t}\Biggl(\sum_{j=0}^{\infty}\beta^{j}\theta^{j} \left(\frac{1-\theta\pi_{t+j}^{\frac{1}{\lambda{p,t+j}}}}{1-\theta} \right)^{1+\lambda_{p,t+j}} \frac{Y_{t+j}\lambda_{t+j}}{\lambda_{p,t+j}} \left[(1+\lambda_{p,t+j})mc_{t+j}\right]\Biggr)}{E_{t}\Biggl(\sum_{j=0}^{\infty}\beta^{j}\theta^{j} \left(\frac{1-\theta\pi_{t+j}^{\frac{1}{\lambda{p,t+j}}}}{1-\theta} \right)^{1+\lambda_{p,t+j}}  \frac{Y_{t+j}\lambda_{t+j}}{\lambda_{p,t+j}P_{t+j}}\Biggr)}
\end{split}
\end{equation}

The way most people seem to solve this is to log-linearize the equation.  This turns the infinite sum into a difference equation.  In addition, the analysis is greatly simplified by assuming that gross inflation ($\pi$) equals 1 in the steady state.  Assuming that gross inflation equals 1 in the SS, I am getting the following equation for the price chosen by firms who can change price in period $t$:

\begin{equation}
\begin{split}
 &   p_{t}^{*}=(1+\lambda_{p})\bar{mc}(1-\beta\theta)E_{t}\left[\sum_{j=0}^{\infty}\beta^{j}\theta^{j}\left[\frac{mc_{t+j}-\bar{mc}}{\bar{mc}}+\frac{P_{t+j}-\bar{P}}{\bar{P}}\right]+\frac{\lambda_{p}}{(1+\lambda_{p})}\sum_{j=0}^{\infty}\beta^{j}\theta^{j}\left(\frac{\lambda_{p,t+j}-\lambda_{p}}{\lambda_{p}}\right)\right]
\end{split}
\end{equation}


Nominal profits from the intermediate-goods producers, $d_{i,t}$ are distributed as dividends to households.


\subsection{Government/Fiscal Authority}

The government is described by a budget constraint and rules fiscal policy rules.  Some, like \cn{Zubairy2010} and \cn{TY2010}, have equations like Taylor rules for fiscal policy, the parameters of which they estimate using historical data.  I don't see the need for this.  I'm thinking we just have a projection of tax law into the future (though we can run into problems if that is not sustainable- i.e. can't have increasing debt forever).  We perturb policies from this for to see the effects (e.g. lower the consumption tax rate by five percentage points for eight quarters, after which is reverts to it's earlier level).  But I can consider other alternatives if needed.

\subsection{Fiscal authority equations}\label{TAppSecFiscAuth}

The government budget constraint represents 1 equation with two unknowns $\{G_t,X_t\}$. It has 8 tax parameters $\{\tau_t^c,\tau_t^l,\tau_t^k,\tau_t^d,\tau_t^{ic},\delta_t^\tau,e_t^\tau,\tau_t^i\}$, 1 spending parameter $\{\gamma_{x,t}\}$, and 13 other endogenous aggregate variables $\{C_t,P_t,L_t,B_t,B_{t-1},w_t,r_t^k,r_t,v_t,K_{t-1},D_t,I_t,K_{t-1}^\tau\}$.
\begin{equation}\label{TAppEqGovtBC}
   \begin{split}
      &\tau^{c}C_t + \frac{\tau_{t}^{l}L_{t}w_{t}}{P_{t}} + \frac{B_{t}}{P_{t}} + \tau_{t}^{k}r_{t}^{k}v_{t}K_{t-1} + \frac{\tau_{t}^{d}D_{t}}{P_{t}} = ... \\
      &\quad\quad\quad\quad G_{t} + X_{t} + \tau_{t}^{ic}I_{t} + \tau_{t}^{k}\delta_{t}^{\tau}K_{t-1}^{\tau} + \tau_{t}^{k}e_{t}^{\tau}I_{t} + \frac{B_{t-1}\left[1+r_{t}(1-\tau_{t}^{i})\right]}{P_{t}}
   \end{split}
\end{equation}
Because $G_t$ and $X_t$ are not independently identified, we must make an assumption about how any changes in tax revenue affect spending on public goods versus spending on transfers. We assume here that spending on public goods and on transfers are each a fixed portion of total revenues. If we solve \eqref{TAppEqGovtBC} for $G_t + X_t$, the other side of the equation represents total revenues. We will assume that $\gamma_{x,t}\in(0,1)$ percent of revenues are spent on transfers and $(1-\gamma_{x,t})$ percent is spent on public goods.
\begin{equation}\label{TAppEqPubGood}
   \begin{split}
      G_t = &(1 - \gamma_{x,t})\Biggl[\tau^{c}C_t + \frac{\tau_{t}^{l}L_{t}w_{t}}{P_{t}} + \frac{B_{t}}{P_{t}} + \tau_{t}^{k}r_{t}^{k}v_{t}K_{t-1} + \frac{\tau_{t}^{d}D_{t}}{P_{t}} ... \\
      &\quad\quad\quad - \tau_{t}^{ic}I_{t} - \tau_{t}^{k}\delta_{t}^{\tau}K_{t-1}^{\tau} - \tau_{t}^{k}e_{t}^{\tau}I_{t} - \frac{B_{t-1}\left[1+r_{t}(1-\tau_{t}^{i})\right]}{P_{t}}\Biggr]
   \end{split}
\end{equation}

\begin{equation}\label{TAppEqTfer}
   \begin{split}
      X_t = &\gamma_{x,t}\Biggl[\tau^{c}C_t + \frac{\tau_{t}^{l}L_{t}w_{t}}{P_{t}} + \frac{B_{t}}{P_{t}} + \tau_{t}^{k}r_{t}^{k}v_{t}K_{t-1} + \frac{\tau_{t}^{d}D_{t}}{P_{t}} ... \\
      &\quad\quad\quad\quad\quad - \tau_{t}^{ic}I_{t} - \tau_{t}^{k}\delta_{t}^{\tau}K_{t-1}^{\tau} - \tau_{t}^{k}e_{t}^{\tau}I_{t} - \frac{B_{t-1}\left[1+r_{t}(1-\tau_{t}^{i})\right]}{P_{t}}\Biggr]
   \end{split}
\end{equation}
Now we have two equations and two unknowns $\{G_t,X_t\}$.



\subsection{Monetary Authority}

The monetary authority follows a Taylor rule, targeting the nominal, after-tax interest rate (I'm taking this from \cn{CER2010}, but feel free to modify to whatever- though I think it's important that we have the zero bound condition):

\begin{equation}
r_{t+1} = max(M_{t+1},0)
\end{equation}

Where $M_{t+1}$ (better notation??) is given by:

\begin{equation}
(1-\tau^{i}_{t})M_{t+1} = \frac{1}{\beta}(1+\pi_{t})^{\phi_{1}}\left(\frac{Y_{t}}{\bar{Y}}\right)^{\phi_{2}}\left(\frac{1+r_{t}}{1+\bar{r}}\right)^{\rho_{r}} - 1
\end{equation}

\begin{equation*}
r_{t+1} = \max\Biggl\{\frac{\frac{1}{\beta}(1+\pi_{t})^{\phi_{1}}\left(\frac{Y_{t}}{\bar{Y}}\right)^{\phi_{2}}\left(\frac{1+r_{t}}{1+\bar{r}}\right)^{\rho_{r}} - 1}{(1-\tau^{i}_{t}},\: 0\Biggr\}
\end{equation*}

Where $\pi_{t}=\frac{P_{t}-P_{t-1}}{P_{t-1}}$ and $\phi_{1} > 1$,  $\phi_{2}\in(0,1)$, and  $\rho_{r}\in(0,1)$.

\subsection{Equilibrium}


Standard definition: prices, wages, and interest rates to clear markets (labor, goods, bond)

Market Clearing:
\begin{itemize}
\item Household demand for bonds equals gov't supply
\item Household labor supply equals intermediate producer labor demand
\item Household capital supply equals intermediate producer capital demand
\item Household and gov't consumption equals supply from the final goods producer
\end{itemize}

\subsection{Solving the model}

Probably best to use some linearization method.  We used the AIM algorithm (\cn{AM1985}) in Willis' class and I still have the code for that.  Alternatively, we used Christiano's method of undetermined coefficients (\cn{Christiano2002}) in Corbae's class.  Chris Sim's has the code for his method (\cn{Sims2002}) available on his website.  \cn{Uhlig1998} does a good job summarizing these methods.

\subsubsection{Characterizing equations}
\label{TAppCharEqs}

Altogether, this model has 36 endogenous variables and 28 exogenous parameters. The 36 endogenous variables are characterized by the 37 equations listed in the sections below. By Walras' Law, one of the market clearing conditions is redundant. In the computation, we discard the resource constraint \eqref{TAppEqMarkClrResConst}.
\begin{align*}
   \text{Endogenous variables:}\quad &\left\{c_t,\lambda_t,l_t,i_t,b_t,v_t,k_t,k_t^\tau,\ve_t^b\right\} \\
   &\left\{Y_t,y_{i,t}^d,P_t,\lambda_{p,t}\right\} \\
   &\left\{y_{i,t},\tilde{k}_{i,t},d_{i,t},r_t^k,w_t,p_{i,t},z_t\right\} \\
   &\left\{G_t,X_t\right\} \\
   &\left\{r_t\right\} \\
   &\left\{p_t,d_t,x_t,g_t,k_{i,t},l_{i,t},C_t,L_t,B_t,K_t,K_t^\tau,D_t,I_t\right\} \\
   \text{Exogenous parameters:}\quad &\left\{\tau_t^c,\tau_t^i,\tau_t^l,\tau_t^k,\tau_t^{ic},\tau_t^d,\delta_t^\tau,e_t^\tau\right\} \\
   &\left\{\beta,\zeta,\sigma\right\} \\
   &\left\{\delta_0,\delta_1,\delta_2,\gamma\right\} \\
   &\left\{\rho_\ve,\mu_\ve,\sigma_\ve\right\} \\
   &\left\{\rho_\lambda,\mu_\lambda,\sigma_\lambda\right\} \\
   &\left\{\alpha,\theta\right\} \\
   &\left\{\gamma_{x,t}\right\} \\
   &\left\{\beta_r,\phi_1,\phi_2,\rho_r\right\}
\end{align*}


There are 19 unknown variables (not counting those from the intermediate goods producers and the shocks, {$\varepsilon_{t}^{b},\lambda_{p,t}$- for which we will have equations that we haven't specified yet) : $i, k, b, c, k^{\tau}, \lambda, g, x, d, P, w, z, \varepsilon^{b}, r, r^{k}, v, l, Y, M$.  These breakdown into the following groups:
\begin{itemize}
\item 4 choice variables (Household): $i,v,l,c$
\item 4 prices: $P,w,r,r^{k}$
\item 2 choice variables (gov't): $g,x$, reallly $2+n$ where $n$ is the number tax instruments
\item 1 choice variable (monetary authority): $M$
\item 2 exogenous state: $z,\varepsilon$
\item 5 endogenous state variable: $k,b,d,k^{\tau},Y$
\end{itemize}

Solve for these unknowns with the following equations:
\begin{itemize}
\item 5 FOCs, household $\implies i,v,l,c,b$
\item 3 from FOCs, intermediate goods producers $\implies r^{k},w,P$ (These FOC's give demand for effective capital and labor.  So together with the HH supply of these factors we can use our market clearing conditions to find the eq'm factor prices.)
\item 1 from the profit function of intermediate goods producers $\implies d$
\item 1 from FOCs, final goods producer $\implies Y$
\item 1 from the zero profit condition for the final goods producer $\implies P$ (This together with the FOC's from the intermediate goods producers gives you the aggregate price index.)
\item 1 Gov't budget constraint ($+n$ fiscal policy rules) $\implies X$, ($+n$ tax rates)
\item 1 Resource constraint $\implies G$
\item 2 laws of motion $\implies k^{\tau},k$
\item 3 exog shock process $\implies z,\varepsilon^{b},\lambda_{p}$
\item 1 Taylor rule for monetary authority $\implies r,M$ (There is the Taylor Rule plus the fact that $r=max(M,0)$ that determines both these.)
\end{itemize}

So we have 19 unknowns and 19 equations.  We should be able to plug this into Dynare and solve the model.

\subsubsection{Market clearing and resource constraint}
\label{TAppSecMCandRC}

The following 14 market clearing conditions close the model. The last equation \eqref{TAppEqMarkClrResConst} is a resource constraint, and can be thought of as goods market clearing. By Walras' Law, one of the following conditions is redundant.

\begin{equation}\label{TAppEqMarkClrC}
   c_t = C_t
\end{equation}

\begin{equation}\label{TAppEqMarkClrL}
   \int_{0}^{1}l_{i,t}di = l_t = L_t
\end{equation}

\begin{equation}\label{TAppEqMarkClrI}
   i_t = I_t
\end{equation}

\begin{equation}\label{TAppEqMarkClrK}
   \int_{0}^{1}\tilde{k}_{i,t}di = v_{t}k_{t} = V_{t}K_{t}
\end{equation}

\begin{equation}\label{TAppEqMarkClrKtau}
   k_t^\tau = K_t^\tau
\end{equation}

\begin{equation}\label{TAppEqMarkClry}
   \left(\int_{0}^{1}y_{i,t}^{\frac{1}{1+\lambda_{p,t}}} di \right) = Y_{t}^{d} = Y_{t}
\end{equation}

\begin{equation}\label{TAppEqMarkClrD}
   \left(\int_{0}^{1}d_{i,t}\right) = d_t = D_t
\end{equation}

\begin{equation}\label{TAppEqMarkClrpit}
   \left(\int_{0}^{1}p_{i,t}^{\frac{-1}{\lambda_{p,t}}}di\right)^{-\lambda_{p,t}} = P_t
\end{equation}

\begin{equation}\label{TAppEqMarkClrG}
   g_t = G_t
\end{equation}

\begin{equation}\label{TAppEqMarkClrX}
   x_t = X_t
\end{equation}

\begin{equation}\label{TAppEqMarkClrResConst}
   Y_t = C_t + I_t + G_t
\end{equation}


\subsection{Calibration/Parameterization}

Many of these DSGE papers estimate the model (e.g., \cn{TY2010}, \cn{Zubairy2010}, \cn{SW2003}), but I think all we need to do is calibrate (or even just parameterize at \cn{CER2010} do.  Not really sure why need to estimate yet another time if other have done it.  Idea is to pick parameters close to other models finding large spending multipliers and show that similar assumptions/parameters also show that certain tax policies also have really large multipliers.  Likely want to calibrate to European economy because want a place with a consumption tax.  UK is probably a good bet since they have their own central bank and a VAT of about 20\%.

\begin{itemize}
\item $\delta_{0}$= annual depreciation rate ($\sim$10\%), equals the aggregate SS investment rate ($\frac{\bar{I}}{\bar{K}}$).
\item $\delta_{1}$ =
\item $\delta_{2}$ =
\item $\delta^{\tau}$ = Not sure how set- varies by asset type and front loaded, but probably just set to 12-14\%- something higher than econ depreciation
\item $\tau^{l}$ = See \cn{Jones2002} or just do statutory rate for median household or use tax calculator to get marginal rate for median household
\item $\tau^{i}$ = assume = labor income tax
\item $\tau^{d}$ = See \cn{Jones2002} or just do statutory rate for median household or use tax calculator to get marginal rate for median household
\item $\tau^{k}$ = See \cn{Jones2002}
\item $\tau^{c}$ = See \cn{Jones2002}
\item $\tau^{ic}$ = 0, most years for most investments this is zero
\item $e^{\tau}$ = 0, in reality expensing for tax $> 0$ for many small businesses at most times, but we can set to zero
\item $\gamma$ =
\item $\zeta$ = I think this is the Frisch elasticity of labor.  \cn{CER2010} set to 0.29.
\item $\sigma$ = coeff of relative risk aversion.  Often set to 2.
\item $\chi^{g}$ =
\item $\sigma_{g}$ =
\item $\sigma_{\ve}$ =
\item $\rho_{\ve}$ =
\item $\sigma_{\lambda}$ =
\item $\lambda_{p}$ = mean markup = $\frac{1+\lambda_{p}}{\lambda_{p}}$, \cn{BF1995} give U.S. evidence of avg markup of 10-14\% which implies $\lambda_{p}\sim 8$.
\item $\beta$ = set to match (real?) after-tax interest rate (given tax on interest income)
\item $\alpha$ = capital share of output (30-35\%)
\item $\sigma_{z}$ = Estimate log of production function using aggregate data (typical business cycle accounting) to get residual.  Fit the residual to AR(1) to get $\sigma_{z}$ and $\rho_{z}$.
\item $\rho_{z}$ = Estimate log of production function using aggregate data (typical business cycle accounting) to get residual.  Fit the residual to AR(1) to get $\sigma_{z}$ and $\rho_{z}$.
\item $\theta$ = fraction of firms changing price
\item $\phi_{1}$ = estimate a log-log specification of the Taylor rule
\item $\phi_{2}$ = estimate a log-log specification of the Taylor rule
\item $\rho_{r}$ = estimate a log-log specification of the Taylor rule
\item $\bar{L}$ = fraction of hours worked ($0.3 \sim \frac{8}{24}$, 0.5 corresponds to Frisch elasticity = 1??)
\item $\frac{\bar{G}}{\bar{Y}}$ = historical average
\item $\frac{\bar{B}}{\bar{Y}}$ = historial average
\item $\frac{\bar{X}}{\bar{Y}}$ = historical average
\end{itemize}

TABLE: Parameritization

\section{Short Run Multipliers}

\begin{enumerate}
\item FIGURES: impulse response functions for tax cuts (capital, income, consumption, investment) and gov't spending increase (\% changes in output, emp, cons, inv, int rates, inflation) to policy change
\item TABLE: table with size of multipliers
\end{enumerate}

\section{Long Run Multipliers}

FIGURE: graph of multiplier from each of fiscal policy measures over time

\section{Sensitivity to Government Financing}

FIGURE: graph of multiplier over time given how gov't financing temporary tax cuts (with higher levels of debt going forward or with paying back by increasing taxes).  Maybe 3D graph with multiplier, years from cut, and year to pay back debt as axes.

\section{Sensitivity to Monetary Policy}

Talk about multipliers at and away from the zero-bound.  What happens if hold interest rates constant?

\section{Dead-weight Loss and Fiscal Policy}

\begin{enumerate}
\item TABLE: table of DWL for each of policy responses
\item FIGURE: graph of DWL over time for policy response
\end{enumerate}

Do some revenue neutral changes in tax policy and calculate welfare....

\section{Sensitivity to Parameters}

How do results change with change to key parameters...


\section{Discussion}

Note that hard to know preferences for gov't spending, but imagine that people know better what want, cutting taxes may be politically easier than increasing spending.  Note other weaknesses of the modeling assumptions.

\section{Conclusion}



\section{Questions:}
\begin{enumerate}
\item Do those with models of the multiplier usually have transfers to households or gov't actually buying stuff?
\item Do I want to say anything about optimal consumption tax being non-zero (like want non-zero inflation) because can more easily do tax cut to respond to recession than could do negative tax rate (spending)?
\item Did Europe (or even US states) have any changes to consumption taxes in response to the recession? \emph{A: I don't think so.  In fact, I believe some states cancelled their sales tax holidays due to budget shortfalls.}
\item Do we want to have rules for fiscal policy as in \cn{Zubairy2010}?  Again, not sure why we need it if not trying to match data.  Just need to estimate SS levels and propose levels that will revert to after aggregate shock.
\end{enumerate}

\section{Does anyone do this?}
\begin{enumerate}
\item \cn{IMF2010} shows impulse response functions to a consumption tax cut.  This paper seems to find that consumption taxes do well compared to other taxes and transfers, but don't do as well as gov't consumption and gov't investment
\item \cn{LPT2010} have consumption taxes, but don't allow them to vary with gov't debt
\item \cn{Zubairy2010} doesn't not have consumption taxes, only capital and labor income taxes.  Says few even have distortionary taxes in models of multipliers.
\item \cn{TY2010} allow for consumption taxes and allow them to vary.  Not answering the same question as I want, but good to look at what they say about consumption taxes.  In particular, how with cons taxes the consumer and producer price index will vary and what crowding out happens with consumption tax changes.
\item \cn{SC2005} have consumption taxes, but don't use them as a policy instrument.  They don't say anything interesting about consumption taxes.
\item \cn{CER2010} don't have consumption taxes and only focus on spending multipliers.  I don't even think they have taxes.  Fiscal and monetary policy rules look simple though- might want to take these.
\end{enumerate}


\section{Links}
\begin{itemize}
\item \url{http://www.econbrowser.com/archives/2010/03/policy_in_dsges.html}
\item \cn{IMF2010} paper: \url{http://www.imf.org/external/pubs/ft/wp/2010/wp1073.pdf}
\item \url{http://delong.typepad.com/sdj/2009/07/cracking-chistiano-eichenbaum-and-rebelos-big-multipliers-without-coffee.html}
\item \url{http://www.econbrowser.com/archives/multipliers/index.html}
\item \cn{TY2010} paper: \url{http://www.nber.org/papers/w15160}
\end{itemize}




\section{MORE QUESTIONS:}
\begin{enumerate}
\item Why do \cn{CER2010} have a subsidy to correct for monopoly of int goods producers?  I don't see others with this.
\item What is capital tax equivalent to?  See what \cn{Zubairy2010} estimates from, but it's like a corp income tax, cap gains tax, div tax rolled into one.
\item Do Multiplier of: gov't spending (baseline to compare), cut in: cap tax, labor tax, cons tax, invest tax credit/bonus deprec/accelerated expensing
\item Can I do bonus deprec with my model- or do I need to keep track of vintage/basis since not partial expensing?
\end{enumerate}

\newpage

\bibliography{TaxMult_bib}



\newpage
\renewcommand{\theequation}{T.\arabic{section}.\arabic{equation}}
                                                 % redefine the command that creates the section number
\renewcommand{\thesection}{T-\arabic{section}}   % redefine the command that creates the equation number
\setcounter{equation}{0}                         % reset counter
\setcounter{section}{0}                          % reset section number
\section*{TECHNICAL APPENDIX}                    % use *-form to suppress numbering

\setcounter{equation}{0} % reset counter
\section{Derivation of monopolistic competition intermediate goods demand and aggregate price Equations}\label{TAppDemPrice}

The profit maximization approach to solving for each intermediate good demand equation $y_{i,t}$ and the aggregate price equation $P_t$ is the following.

\begin{equation}
\max_{y_{i,t}} P_{t}Y_{t}^{d} - \int_{0}^{1}p_{i,t}y_{i,t}di
\end{equation}

The FOC's are:

\begin{equation}
p_{t}(1+\lambda_{p,t})\left(\int_{0}^{1}y_{i,t}^{\left(\frac{1}{1+\lambda_{p,t}}\right)}di\right)^{\lambda_{p,t}}\left(\frac{1}{1+\lambda_{p,t}}\right)y_{i,t}^{\frac{1}{1+\lambda_{p,t}}-1}-p_{i,t} = 0, \forall i
\end{equation}

Dividing the FOCs for intermediate inputs $i$ and $j$ $\implies$
\begin{equation}
\frac{p_{i,t}}{p_{j,t}} = \left(\frac{y_{i,t}}{y_{j,t}}\right)^{\frac{1}{1+\lambda_{p,t}}-1} = \left(\frac{y_{i,t}}{y_{j,t}}\right)^{\frac{-\lambda_{p,t}}{1+\lambda_{p,t}}}
\end{equation}

Rearranging:
\begin{equation}
\begin{split}
&\implies p_{i,t} = \left(\frac{y_{j,t}}{y_{i,t}}\right)^{\frac{\lambda_{p,t}}{1+\lambda_{p,t}}}p_{j,t} \\
& \implies p_{i,t} = y_{j,t}^{\frac{\lambda_{p,t}}{1+\lambda_{p,t}}}p_{j,t}y_{i,t}^{\frac{-\lambda_{p,t}}{1+\lambda_{p,t}}} \\
& \implies p_{i,t}y_{i,t} = p_{j,t}y_{j,t}^{\left(\frac{\lambda_{p,t}}{1+\lambda_{p,t}}\right)}y_{i,t}^{\frac{1}{1+\lambda_{p,t}}}
\end{split}
\end{equation}

Then integrate this condition to yield:
\begin{equation}
\begin{split}
\int_{0}^{1}p_{i,t}y_{i,t}di & = p_{j,t}y_{j,t}^{\left(\frac{\lambda_{p,t}}{1+\lambda_{p,t}}\right)}\int_{0}^{1}y_{i,t}^{\frac{1}{1+\lambda_{p,t}}}di  \\
& = p_{j,t}y_{j,t}^{\left(\frac{\lambda_{p,t}}{1+\lambda_{p,t}}\right)}(Y_{t})^{\frac{1}{1+\lambda_{p,t}}}
\end{split}
\end{equation}

The zero profit condition implies $P_{t}Y_{t}=\int_{0}^{1}p_{i,t}y_{i,t}di$.  Plugging this into the above and we find:
\begin{equation}
\begin{split}
P_{t}Y_{t} &= p_{j,t}y_{j,t}^{\left(\frac{\lambda_{p,t}}{1+\lambda_{p,t}}\right)}(Y_{t})^{\frac{1}{1+\lambda_{p,t}}} \\
\implies &P_{t}Y_{t}=p_{j,t}y_{j,t}^{\frac{\lambda_{p,t}}{1+\lambda_{p,t}}}(Y_{t})^{\frac{1}{1+\lambda_{p,t}}} \\
\implies & P_{t} =  p_{j,t}y_{j,t}^{\frac{\lambda_{p,t}}{1+\lambda_{p,t}}}(Y_{t})^{\frac{-\lambda_{p,t}}{1+\lambda_{p,t}}}
\end{split}
\end{equation}

Which implies the demand function (just rearranging terms):
\begin{equation}
\begin{split}
y_{i,t} =  \left(\frac{p_{i,t}}{P_{t}}\right)^{\frac{-(1+\lambda_{p,t})}{\lambda_{p,t}}}Y_{t}, \forall i
\end{split}
\end{equation}

To find $P_{t}$ use the zero profit condition:
\begin{equation}
\begin{split}
& P_{t}Y_{t} = \int_{0}^{1}p_{i,t}y_{i,t}^{\left(\frac{\lambda_{p,t}}{1+\lambda_{p,t}}\right)}(Y_{t})^{\frac{1}{1+\lambda_{p,t}}}di \\
\implies & P_{t}Y_{t} = \int_{0}^{1}p_{i,t}\left(\left(\frac{p_{i,t}}{P_{t}}\right)^{\frac{-(1+\lambda_{p,t})}{\lambda_{p,t}}}Y_{t}\right)^{\left(\frac{\lambda_{p,t}}{1+\lambda_{p,t}}\right)}(Y_{t})^{\frac{1}{1+\lambda_{p,t}}}di \\
\implies & P_{t}Y_{t} = \int_{0}^{1}p_{i,t}\left(\frac{p_{i,t}}{P_{t}}\right)^{\frac{-(1+\lambda_{p,t})}{\lambda_{p,t}}}Y_{t}di \\
\implies & P_{t}Y_{t} = \int_{0}^{1}p_{i,t}p_{i,t}^{\frac{-(1+\lambda_{p,t})}{\lambda_{p,t}}}P_{t}^{\frac{(1+\lambda_{p,t})}{\lambda_{p,t}}}Y_{t}di \\
\implies & P_{t}Y_{t} = \int_{0}^{1}p_{i,t}^{\frac{-1}{\lambda_{p,t}}}di \:  Y_{t}P_{t}^{\frac{1+\lambda_{p,t}}{\lambda_{p,t}}} \\
\implies & P_{t} = \int_{0}^{1}p_{i,t}^{\frac{-1}{\lambda_{p,t}}}di \: P_{t}^{\frac{1+\lambda_{p,t}}{\lambda_{p,t}}} \\
\implies & P_{t}^{\frac{-1}{\lambda_{p,t}}} = \int_{0}^{1}p_{i,t}^{\frac{-1}{\lambda_{p,t}}}di \\
\implies & P_{t} = \left(\int_{0}^{1}p_{i,t}^{\frac{-1}{\lambda_{p,t}}}di\right)^{-\lambda_{p,t}} \\
\end{split}
\end{equation}

The cost minimization approach to solving for each intermediate good demand equation $y_{i,t}$ and the aggregate price equation $P_t$ is the following.
\begin{equation}\label{TAppEqObjCostMin}
   \min_{y_{i,t}} \int_{0}^{1}p_{i,t}y_{i,t}di \quad\text{s.t.}\quad Y_t \leq \left(\int_{0}^{1} y_{i,t}^{\frac{1}{1+\lambda_{p,t}}}di\right)^{1+\lambda_{p,t}}
\end{equation}
The Lagrangian for this minimization problem is the following,
\begin{equation}\label{TAppEqLagrCostMin}
   \mathcal{L} = \int_{0}^{1}p_{i,t}y_{i,t}di + \lambda_y\left[Y_t - \left(\int_{0}^{1} y_{i,t}^{\frac{1}{1+\lambda_{p,t}}}di\right)^{1+\lambda_{p,t}}\right]
\end{equation}
in which the multiplier $\lambda_y$ has the interpretation of being the marginal cost of an extra unit of aggregated output. That is, $\lambda_y$ is the price of aggregate output $P_t$.
\begin{equation}\label{TAppEqLagrCostMin2}
   \mathcal{L} = \int_{0}^{1}p_{i,t}y_{i,t}di + P_t\left[Y_t - \left(\int_{0}^{1} y_{i,t}^{\frac{1}{1+\lambda_{p,t}}}di\right)^{1+\lambda_{p,t}}\right]
\end{equation}
We need to finish this....


\newpage
\setcounter{equation}{0} % reset counter
\section{Derivation of real marginal costs}\label{TAppRealMC}

Dividing the period profits equation \eqref{EqIntGoodDiv} by aggregate prices $P_t$ gives the following real total costs function.
\begin{equation}\label{TAppEqRealTC}
   tc_t = \frac{r_t^k}{P_t}\tilde{k}_{i,t} + \frac{w_t}{P_t}l_{i,t}
\end{equation}
Substituting in the expression for $\tilde{k}_{i,t}$ from the capital-labor ratio equation \eqref{Eqklratio} gives real total costs in terms of $\alpha$, $w_t$, $P_t$, and $l_{i,t}$.
\begin{equation}\label{TAppEqRealTCl}
   tc_t = \frac{1}{1-\alpha}\left(\frac{w_t}{P_t}\right)l_{i,t}
\end{equation}

Because the intermediate goods production exhibits constant returns to scale, we can find the real marginal cost $mc_t$ by finding the amount of labor necessary to produce 1 unit of output, and substituting that value into the real total costs equation \eqref{TAppEqRealTCl}.
\begin{equation}\label{TAppEqlunitprod1}
   z_t\tilde{k}_{i,t}^\alpha l_{i,t}^{1-\alpha} = 1
\end{equation}
Next, substitute in the equation for capital $\tilde{k}_{i,t}$ in terms of labor from the capital-labor ratio equation \eqref{Eqklratio}.
\begin{equation}\label{TAppEqlunitprod2}
   z_t\left(\frac{\alpha}{1-\alpha}\left[\frac{w_t}{r_t^k}\right]l_{i,t}\right)^\alpha l_{i,t}^{1-\alpha} = 1 \quad\Rightarrow\quad l_{i,t} = \frac{1}{z_t}\left(\frac{\alpha}{1-\alpha}\left[\frac{w_t}{r_t^k}\right]\right)^{-\alpha}
\end{equation}

Lastly, substitute the amount of labor necessary to produce one unit of output from \eqref{TAppEqlunitprod2} into the expression for real total costs \eqref{TAppEqRealTCl} to get real marginal costs.
\begin{equation}\label{TAppEqRealMC}
   mc_t = \frac{1}{1-\alpha}\left(\frac{w_t}{P_t}\right)\frac{1}{z_t}\left(\frac{\alpha}{1-\alpha}\left[\frac{w_t}{r_t^k}\right]\right)^{-\alpha} = \frac{w_t^{1-\alpha}(r_t^k)^\alpha}{z_t P_t \alpha^\alpha (1-\alpha)^{1-\alpha}}
\end{equation}




\end{document}
